% !TEX TS-program = xelatex
% !TEX encoding = UTF-8 Unicode
% !Mode:: "TeX:UTF-8"

\documentclass{resume}
\usepackage{zh_CN-Adobefonts_external} % Simplified Chinese Support using external fonts (./fonts/zh_CN-Adobe/)
% \usepackage{NotoSansSC_external}
% \usepackage{NotoSerifCJKsc_external}
% \usepackage{zh_CN-Adobefonts_internal} % Simplified Chinese Support using system fonts
\usepackage{linespacing_fix} % disable extra space before next section
\usepackage{cite}

\begin{document}
\pagenumbering{gobble} % suppress displaying page number

\name{张星}

\basicInfo{
  \email{zhangxing610321@gmail.com} \textperiodcentered\
  \phone{(+86) 17681826407} \textperiodcentered\
  \github[acton393]{http://github.com/acton393}
}

\section{\faGraduationCap\  教育背景}
\datedsubsection{\textbf{西安电子科技大学}, 西安}{2011 -- 2015}
\textit{学士 学位}\ 计算机科学与技术

\textit{{CET6}: 516}

% Reference Test
%\datedsubsection{\textbf{Paper Title\cite{zaharia2012resilient}}}{May. 2015}
%An xxx optimized for xxx\cite{verma2015large}
%\begin{itemize}
%  \item main contribution
%\end{itemize}

\section{\faCogs\ 技能}
% increase linespacing [parsep=0.5ex]
\begin{itemize}[parsep=0.5ex]
  \item 了解业界 ReactNative、Weex 和 Flutter 等架构
  \item 掌握 TypeScript、JavaScript、Objective-C、C/C++,熟悉常见的数据结构和算法
  \item 了解 GN、CMake 等构建工具,了解 JNI 开发
  \item 了解 hybrid,熟悉 iOS 平台的开发工具及开发
  \item 开源社区贡献者,同时也是 Apache commiter
\end{itemize}

\section{\faUsers\ 工作经历}
\datedsubsection{\textbf{杭州微米网络有限公司}\ 上海}{2019年3月 -- 今}
\role{资深开发工程师}\

\subsubsection*{{拼多多客户端动态化项目乐高基础建设}}
\role{TypeScript,Objective-C/C++}{从0-1建设面向大前端的动态化开发框架,不依赖 JavaScript 的执行}
\begin{onehalfspacing}
  \begin{itemize}
    \item iOS 端框架搭建,负责动画,文字,图片,canvas,dom 等核心能力的建设以及性能优化
    \item 双端 Layout engine 优化接入
    \item 负责稳定性,可用性
    \item 表达式标准函数库,Test 262 case(ECMAScript Test suit)支持
  \end{itemize}
\end{onehalfspacing}
\subitem
\subsubsection*{九块九项目 owner}
\role{TypeScript}{运用乐高动态化能力落地业务}
\begin{onehalfspacing}
  \begin{itemize}
    \item 设计后端数据驱动前端模板展示
    \item 保持业务高效快速迭代
  \end{itemize}
\end{onehalfspacing}


\datedsubsection{\textbf{阿里巴巴中国有限公司}\ 浙江杭州}{2015年7月 -- 2019年3月}
\role{资深开发工程师}\
主要职责:原生渲染,JSCore 和 Native 通信优化,富交互能力的输出,双十一等大促活动稳定性和性能的保障。
\datedsubsection{\textbf{Weex JavaScript 执行环境的沙盒隔离}\ {2018年1月--2018年4月}}\
\role{职责: 项目 iOS 开发和方案设计,推动其他端参与,各个业务方的验收和回归}\
\textit{项目背景:Weex很长一段时间都是各个业务方的页面逻辑都在一个 JS 环境中执行,后一个页面的渲染执行效果很大程度上受到上一个页面的影响。全局污染,js内存长期居高不下等问题都是由于共享执行环境上下文导致。}\
\textit{解决方案: 为所有 Weex 运行的页面(实例)创建新的环境,在 SDK 初始化完成之后创建全局实例作为 manager, 业务页面的环境由 manager 来初始化和提供。}

\begin{itemize}
  \item js 报错能精确定位到页面。
  \item 进出页面几次后内存居高不下的情况得到了控制,反复进出页面后持有的内存都能得到合理的释放。
\end{itemize}


\datedsubsection{\textbf{RecycleList 长列表组件项目}\ {2017年10月--2017年1月}}\
\role{职责: 项目 iOS 开发和方案设计,推动其他端参与}\
\textit{项目背景:使用 Weex 开发无限列表,随着用户无限的滑动,js 和 native 需要发生频繁的通信,进行动态的加载页面,虚拟 dom 和 native 端的 dom 进行同步,越来越多的内存被占用最终会 OOM 发生abort,这种场景 在低端机上发生的概率更大。对于电商类应用,无限列表这种场景应用很普遍。}\
\textit{解决方案: 尝试在列表初始化的时候把初始化的数据、表达式和事件等一次性传递到 native 侧,js 端不再进行展开,减少虚拟 dom 对内存的占用,同时 native 侧永远只展开屏幕中保留的部分,实施懒加载的方式。如果在 iOS 上如果采用prefetch方案,iOS 会提前预加载组件,体验效果会更好。}

采用 3000 个节点进行压力测试,关键性能数据如下
\begin{itemize}
  \item 页面加载首屏时间基本和之前列表组件数据持平
  \item 下拉到底部占用的内存近似为原来的 1/2
  \item 加载过程中峰值 CPU 为原来 1/5,滑动 CPU 与原来数据持平
\end{itemize}

\datedsubsection{\textbf{Weex 无障碍能力支持}\ {2017年6月--2017年8月}}\
\role{职责: 无障碍能力支持方案设计}\
\textit{Weex 采用绘制的方式较多,一些原生的无障碍的能力严重缺失,如果一个 app 重度依赖这种缺乏无障碍支持的动态化方案,对于一个有障碍的用户来讲,这个 app 基本不能再使用}

\textit{参照 W3C Web 对于无障碍能力的支持设计,完成 iOS 无障碍支持的开发,同时推动 Android 也补齐对应缺失的能力,尽量向 web 端对齐,同时暴露原生的一些无障碍的能力}



\datedsubsection{\textbf{Weex 文字组件渲染}\ {2017年1月--2017年4月}}\
\role{职责: iOS 开发和方案设计,推动各个业务方的验收和回归}\
\textit{项目背景: Weex 原有文字采用 textkit 来绘制,在 Weex 这种多线程的场景下会有比较高的概率发生 crash,crash堆栈都是在 textkit 中,使用 textkit 存在一些文字无法对齐的场景}

textit{采用比textkit 更底层的方案coreText来绘制文字,获得组件的排版和布局之后,对于一行文字的行高和基线都可控,对于多线程绘制的地方尽量避免多个线程同时去读写一个对象}


\datedsubsection{\textbf{腾讯(深圳)有限公司}\  深圳}{2014年6月 -- 2014年10月}
\role{实习服务端开发工程师}\
在 TEG 团队提供运维管理平台后端服务。


\section{\faInfo\ 自我评价}
% increase linespacing [parsep=0.5ex]

\textit{
  性格开朗、思维活跃,待人真诚,做事有责任心、可靠,条理性、责任心强;易与人相处,对生活充满热情,任劳任怨,勤奋好学,敢挑重担,具有比较强的团队精神和协调能力,业余喜欢篮球、音乐,也会读书、了解时政、科技。
}


%% Reference
%\newpage
%\bibliographystyle{IEEETran}
%\bibliography{mycite}
\end{document}
